\documentclass[12pt]{article}
\usepackage[utf8]{inputenc} % obligatoire pour Overleaf
\usepackage{url}

\begin{document}
Dix secrets pour réussir un bon exposé scientifique

Par : Vous

Introduction

Le texte de cet exercice est une version significativement abrégée et légèrement modifié
de l'excellent article ainsi intitulé par Mark Schoeberl et Brian Toon :
\url{http://www.cgd.ucar.edu/cms/agu/scientific_talk.html}

Les secrets

J'ai compilé cette liste personnelle de "secrets" en écoutant des orateurs efficaces et inefficaces. Je ne prétends pas que cette liste est exhaustive - je suis sûr qu'il y a des choses que j'ai oubliées. Mais ma liste comporte probablement 90% de ce que vous devez savoir et faire.

1) Préparez votre matériel attentivement et avec logique. Racontez une histoire.

2) Faites des répétitions. Il n'y pas d'excuse pour un manque de préparation.

3) N'y insérez pas trop de matière. Les bons orateurs choisissent un ou deux points importants et restent là-dessus.

4) Évitez les équations. On dit que pour chaque équation dans votre présentation, le nombre d'auditeurs qui vont comprendre sera divisé par deux. Autrement dit, si q est le nombre d'équations dans votre présentation et n le nombre de personnes qui la comprennent, alors 

n = gamma (1/2) à la puissance q

où gamma est une constante de proportionnalité.

5) Ne gardez que peu de points de conclusion. Les gens ne peuvent se souvenir que de très peu de choses surtout si elles suivent plusieurs exposés dans des grandes réunions.

6) Adressez-vous à votre auditoire et pas à l'écran. Un des problèmes les plus fréquents est que l'orateur s'adresse à l'écran de projection.

7) Évitez les sons distrayants. Évitez les "Hummm" ou "Ahhh" entre deux phrases.

8) Soignez vos graphiques. Voici une liste de tuyaux pour les bons graphiques :

* Utilisez des gros caractères.

* Gardez les graphiques simples. Ne montrer pas les graphiques dont vous n'avez pas besoin.

* Utilisez la couleur.

9) Soyez charmant quand vous répondez aux questions.

10) Utilisez l'humour si possible. Je suis toujours à quel point une blague même lamentable peut déclencher un bon rire lors d'un exposé scientifique.

\end{document}
