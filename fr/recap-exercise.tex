\documentclass[12pt]{article}

\usepackage{url}

\begin{document}
Dix secrets pour r�ussir un bon expos� scientifique

Par : Vous

Introduction

Le texte de cet exercice est une version significativement abr�g�e et l�g�rement modifi�
de l'excellent article ainsi intitul� par Mark Schoeberl et Brian Toon :
\url{http://www.cgd.ucar.edu/cms/agu/scientific_talk.html}

Les secrets

J'ai compil� cette liste personnelle de "secrets" en �coutant des orateurs efficaces et inefficaces. Je ne pr�tends pas que cette liste est exhaustive - je suis s�r qu'il y a des choses que j'ai oubli�es. Mais ma liste comporte probablement 90% de ce que vous devez savoir et faire.

1) Pr�parez votre mat�riel attentivement et avec logique. Racontez une histoire.

2) Faites des r�p�titions. Il n'y pas d'excuse pour un manque de pr�paration.

3) N'y ins�rez pas trop de mati�re. Les bons orateurs choisissent un ou deux points importants et restent l�-dessus.

4) �vitez les �quations. On dit que pour chaque �quation dans votre pr�sentation, le nombre d'auditeurs qui vont comprendre sera divis� par deux. Autrement dit, si q est le nombre d'�quations dans votre pr�sentation et n le nombre de personnes qui la comprennent, alors 

n = gamma (1/2) � la puissance q

o� gamma est une constante de proportionnalit�.

5) Ne gardez que peu de points de conclusion. Les gens ne peuvent se souvenir que de tr�s peu de choses surtout si elles suivent plusieurs expos�s dans des grandes r�unions.

6) Adressez-vous � votre auditoire et pas � l'�cran. Un des probl�mes les plus fr�quents est que l'orateur s'adresse � l'�cran de projection.

7) �vitez les sons distrayants. �vitez les "Hummm" ou "Ahhh" entre deux phrases.

8) Soignez vos graphiques. Voici une liste de tuyaux pour les bons graphiques :

* Utilisez des gros caract�res.

* Gardez les graphiques simples. Ne montrer pas les graphiques dont vous n'avez pas besoin.

* Utilisez la couleur.

9) Soyez charmant quand vous r�pondez aux questions.

10) Utilisez l'humour si possible. Je suis toujours � quel point une blague m�me lamentable peut d�clencher un bon rire lors d'un expos� scientifique.

\end{document}
